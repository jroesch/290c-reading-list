\begin{pgroup}{Basic Logic Programming}
  \begin{paper}{Pearce}
    \mustread
  \end{paper}

  \begin{paper}{Lloyd}
    Contains a whole lot of background on logic programming, with formal mathematical definitions.
    Extremely detailed, though its very dry and dense so it is not particularly easy to read.
    Also, I cannot find a PDF version of this.

    \mustread
  \end{paper}

  \begin{paper}{Nilsson}
    The first 100 pages or so seem very relevant to the theoretical foundations of Prolog, and it seems to overlap somewhat with~\cite{Lloyd} here.

    \mustread
  \end{paper}

  \begin{paper}{Miller89}
    Discusses how to incorporate a module-like system into a pure LP system.
    This is in constrast to how modules are discussed in the Prolog standard, which are ludicrously ad-hoc and problematic.
    \mustread
  \end{paper}
\end{pgroup}

\begin{pgroup}{Intuitionistic Logic Programming}
  \begin{paper}{McCarty}
    Discusses an implementation of ILP.
    \mustread
  \end{paper}

  \begin{paper}{Bonner}
    Discuss adding negation-as-failure to ILP, from a theoretical standpoint.
    \mustread
  \end{paper}

  \begin{paper}{McRobbie}
    Related to the resolution operator for ILP.
    Seems to be more on the theoretical side.
    \mustread
  \end{paper}

  \begin{paper}{Hui-Bon-Hoa}
    Related to the resolution operator for ILP.
    Seems to be more on the theoretical side.
    \mustread
  \end{paper}

  \begin{paper}{Felty}
    Shows that if we have a higher-order ILP language, we can easily implement theorem provers and even custom search strategies.
    The first three sections also provide good background motivation on why ILP is useful, since the proof rules shown require intuitionistic implication to exist at the metalanguage level to be easily implemented.
    \mustread
  \end{paper}

  \begin{paper}{Bonner88}
    Argues that intuitionistic logic is the logic of hypothetical reasoning, as opposed to classical logic.
    \mustread
  \end{paper}

  \begin{paper}{Gabbay}
    An ILP engine.
    \mustread
  \end{paper}
\end{pgroup}

\begin{pgroup}{Intuitionistic Linear Logic Programing}
  \begin{paper}{Hodas}
    \mustread
  \end{paper}
\end{pgroup}

\begin{pgroup}{Linear Logic Programming}
  \begin{paper}{Winikoff}
    \mustread
  \end{paper}
\end{pgroup}
