\begin{pgroup}{Basic Logic Programming}
  \begin{paper}{Pearce}
    \mustread
  \end{paper}

  \begin{paper}{Lloyd}
    Contains a whole lot of background on logic programming, with formal mathematical definitions.
    Extremely detailed, though its very dry and dense so it is not particularly easy to read.
    Also, I cannot find a PDF version of this.

    \mustread
  \end{paper}

  \begin{paper}{Nilsson}
    The first 100 pages or so seem very relevant to the theoretical foundations of Prolog, and it seems to overlap somewhat with~\cite{Lloyd} here.

    \mustread
  \end{paper}

  \begin{paper}{Miller89}
    Discusses how to incorporate a module-like system into a pure LP system.
    This is in constrast to how modules are discussed in the Prolog standard, which are ludicrously ad-hoc and problematic.
    \mustread
  \end{paper}
\end{pgroup}

\begin{pgroup}{Intuitionistic Logic Programming}
  \begin{paper}{McCarty}
    Discusses an implementation of ILP.
    \mustread
  \end{paper}

  \begin{paper}{Bonner}
    Discuss adding negation-as-failure to ILP, from a theoretical standpoint.
    \mustread
  \end{paper}

  \begin{paper}{McRobbie}
    Related to the resolution operator for ILP.
    Seems to be more on the theoretical side.
    \mustread
  \end{paper}

  \begin{paper}{Hui-Bon-Hoa}
    Related to the resolution operator for ILP.
    Seems to be more on the theoretical side.
    \mustread
  \end{paper}
\end{pgroup}

\begin{pgroup}{Intuitionistic Linear Logic Programing}
  \begin{paper}{Hodas}
    \mustread
  \end{paper}
\end{pgroup}

\begin{pgroup}{Linear Logic Programming}
  \begin{paper}{Winikoff}
    \mustread
  \end{paper}
\end{pgroup}
